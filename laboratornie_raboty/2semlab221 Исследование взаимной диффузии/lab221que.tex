\documentclass[a4paper, fontsize = 14pt]{article}
\usepackage{hyperref}
\usepackage[warn]{mathtext}
\usepackage[english,russian]{babel}
\usepackage[utf8x]{inputenc} 
 
%математика
\usepackage[mathscr]{eucal}
\usepackage{amsmath,amsfonts,amssymb,amsthm,mathtools}
\usepackage{icomma}
\usepackage{wasysym}
\usepackage{mathrsfs}
 
%оформление текста
\usepackage{setspace}
\onehalfspacing
\usepackage{indentfirst}
\usepackage{scrextend}
 
%геометрия
\usepackage{geometry}
\geometry{left=25mm,right=25mm,
 top=25mm,bottom=30mm}
 
%графика
\usepackage{wrapfig}
\usepackage{graphicx}
\usepackage{pgfplots}
\usepackage{tikz}
\usepackage{hvfloat}
\RequirePackage{caption}
\DeclareCaptionLabelSeparator{defffis}{ --- }
\captionsetup{justification=centering,labelsep=defffis}
 
%таблицы
\usepackage{array,tabularx,tabulary,booktabs} 
\usepackage{longtable}  
\usepackage{multirow} 
 
%ссылки
\usepackage{hyperref}
\usepackage{xcolor}
\definecolor{grn}{HTML}{57A14F} %зеленый
\definecolor{rd}{HTML}{E53C44} %красный 
\definecolor{bl}{HTML}{282691} %синий 
\definecolor{bbl}{HTML}{001B6C} %темно-синий
\hypersetup{		
    colorlinks=true,       	
    linkcolor=bbl,          % внутренние ссылки
    citecolor=rd,          % на библиографию
    filecolor=magenta,      % на файлы
    urlcolor=bl           %внешние источники
}
 
% Колонтитулы
\usepackage{fancyhdr} 
 	\pagestyle{fancy}
 	\renewcommand{\headrulewidth}{0.15mm}  
 	\renewcommand{\footrulewidth}{0.15mm}
 	\lfoot{МФТИ, 2021}
 	\rfoot{\thepage}
 	\cfoot{}
 	\rhead{}
 	\chead{}
 	\lhead{}
 
 
\begin{document}

\textbf{1. Диффузия, коэффициент диффузии D.}\\
Диффузией называется перемешивание компонентов смеси, возникающее при наличии перепада их концентраций. На молекулярном уровне причиной диффузии является хаотичное движение
индивидуальных частиц, которое приводит к возникновению направленного течения компонентов смеси. Процесс диффузии направлен к установлению равновесия, то есть к выравниванию концентраций и равномерному перемешиванию компонентов.\\

Исходя из эксперементальных данных можно записать закон Фика, который определяет плотность потока вещества \textbf{i} - количество частиц, перенесенное через единицу площади, перпендикулярной потоку, за единицу времени:
\begin{equation}
i = - D\frac{\partial n}{\partial x},
\end{equation}\\
где D - коэффициент диффузии, n - концентрация вещества. \\
\\
Рассмотрим диффузию легкой примеси, когда масса молекул примеси меньше массы исходного газа, для избежания явления термодиффузии считаем температуру постоянной. Концентрация фонового газа остается постоянной, концентрация примеси зависит от координаты: n(x).\\
Для начала получим выражение для односторонней плотности потока частиц примеси. Будем считать, что все молекулы имеют одинаковую скорость v и движутся параллельно введенным координатным осям трехмерного пространства. \\
Тогда число молекул, пересекающих единичную площадку S сверху вниз(!) в единицу времени будет задавать одностороннюю плотность потока частиц и равняеться:
$$j = \frac{1}{6} Svt n = \frac{1}{6}vn$$ \\
\\
Теперь рассмотрим одномерную задачу.Оценим величину потока примеси через некоторую плоскость, перпендикулярную оси, поместив её для определённости в точку x = 0 (Рис.8). Рассматриваем только те частицы, которые испытали соударение на расстояниях $\pm \lambda$ - порядка длины собственного пробега, иначе вероятность достижения плоскости слишком мала.
\\ 
В данном случае сумма потоков не обнуляется ввиду зависимости концентрации от координаты. $j_{+} =\frac{1}{6} v n$, где n(x) меняется от $x = -\lambda$, аналогично $j_{+} = \frac{1}{6} v n$, где n(x) меняется от $x = + \lambda$ \\
Тогда суммарный поток частиц через плоскость будет равен по порядку величины следующему выражению:
$$j \sim \frac{1}{6}v(n(\lambda) - n(-\lambda))$$
В силу малого изменения концентрации на расстоянии, совпадающему по величине с длиной пробега, можно разложить $n(\pm \lambda)$ в формулу Тейлора в окрестности нуля: 
$$n(\pm \lambda) \approx n(o) \pm\lambda\frac{\partial n}{\partial x}$$ \\
Тогда общая оценка потока :
$$j \sim - \frac{1}{3}v\lambda \frac{\partial n}{\partial x}$$ \\
Отсюда коэффициент диффузии D :
\begin{equation}
D \sim \frac{1}{3}v \lambda  
\end{equation} \\
\\


\end{document}